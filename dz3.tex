\documentclass[a4paper,12pt]{article}

%%% Работа с русским языком
\usepackage{cmap}					% поиск в PDF
\usepackage{mathtext} 				% русские буквы в формулах
\usepackage[T2A]{fontenc}			% кодировка
\usepackage[utf8]{inputenc}			% кодировка исходного текста
\usepackage[english,russian]{babel}	% локализация и переносы


\usepackage[notransparent]{svg}     % svg
% Использование симовла лямюда в рисунках svg 
\usepackage[utf8]{inputenc}
\DeclareUnicodeCharacter{03BB}{$\lambda$}

%% Шрифты
\usepackage{euscript} % Шрифт Евклид
\usepackage{mathrsfs} % Красивый матшрифт



\usepackage[margin=3cm]{geometry}
\usepackage{paralist}
\usepackage{amsthm, amsmath, amsfonts, amssymb}
\usepackage{mathtools} % \mathclap
\usepackage{bm}
\usepackage{dsfont}
\usepackage{hyperref}
\usepackage{graphicx}
\usepackage{multirow}
\usepackage{comment}
\usepackage{xcolor, colortbl}
\usepackage{xifthen, xspace}
\usepackage{caption, subcaption}
\usepackage{lscape}
\usepackage{braket}
\usepackage{epigraph}
\usepackage{listings}


\usepackage{verbatim}

\hypersetup{
    colorlinks=true,
    linkcolor=blue,
    filecolor=magenta,      
    urlcolor=blue,
    pdftitle={Overleaf Example},
    pdfpagemode=FullScreen,
}

%%% Заголовок
\author{Новичихин И.В.}
\title{ДЗ №3: Машины Тьюринга и квантовые вычисления}
\date{\today}

\begin{document}

\maketitle
\newpage






\section{Машины Тьюринга}

Работу требуется выполнять в системе \url{turingmachine.io}. \\\\
Для сдачи заданий 1-2 требуется прикрепить файлы YAML с исходным кодом проекта. Каждый файлы должен иметь наименование задание\_пункт.yml, к примеру 1\_1.yml для первой задачи первого задания. \\\\


% 1 задание МТ
\subsection{Операции с числами}

Реализуйте машины Тьюринга, которые позволяют выполнять следующие операции:

% 1.1
\subsubsection{Сложение двух унарных чисел (1 балл)}

\textbf{Решение представлено в файле \href{https://github.com/NRU-MPEI-IMAI/tm-and-qc-NovichikhinIV/blob/main/yaml/1_1.yaml}{1\_1.yaml}.} \\

Алгортим: заменяем "+" на 1, и удаляем последнюю единицу


% 1.2
\subsubsection{Умножение унарных чисел (1 балл)}

\textbf{Решение представлено в файле \href{https://github.com/NRU-MPEI-IMAI/tm-and-qc-NovichikhinIV/blob/main/yaml/1_2.yaml}{1\_2.yaml}.} \\

Алгоритм: пишем справа знак равно и переносим единицы по одной. Уже перенесенную единицу отмечаем симолом x, после того как всё перенесли стираем то что слева от равно и само равно \\\\






% 2 задание МТ
\subsection{Операции с языками и символами}

Реализуйте машины Тьюринга, которые позволяют выполнять следующие операции:

% 2.1
\subsubsection{Принадлежность к языку $L = \{ 0^n1^n2^n \}, n \ge 0$ (0.5 балла)}

\textbf{Решение представлено в файле \href{https://github.com/NRU-MPEI-IMAI/tm-and-qc-NovichikhinIV/blob/main/yaml/2_1.yaml}{2\_1.yaml}.} \\

На вход поступает строка из 0, 1, 2. Далее по очереди заменяем 0, 1, 2 на x. Если в результате всё заменилось на x, то результат работы y (слово принадлежит языку L). Если нарушится порядок символов или их будет не одинаковое количество, то результат работы n (слово не принадлежит языку L)


% 2.2
\subsubsection{Проверка соблюдения правильности скобок в строке (минимум 3 вида скобок) (0.5 балла)}

\textbf{Решение представлено в файле \href{https://github.com/NRU-MPEI-IMAI/tm-and-qc-NovichikhinIV/blob/main/yaml/2_2.yaml}{2\_2.yaml}.} \\

На вход поступает строка их скобок (), [], \{\}, Будем идти по слову и искать первую закрывающую скобку. Как только она найдена, заменяем скобку на x и идем направо искать открывающую такого же вида (идем направо пока есть символ x), иначе слово не принадлежит языку. Если в итоге в строке остались только символы x, то стираем все символы записываем y, это будет результат программы (слово принадлежит языку). Если не принадлежит языку результат программы символ n

% 2.3
\subsubsection{Поиск минимального по длине слова в строке (слова состоят из символов 1 и 0 и разделены пробелом) (1 балл)}

\textbf{Решение представлено в файле \href{https://github.com/NRU-MPEI-IMAI/tm-and-qc-NovichikhinIV/blob/main/yaml/2_3.yaml}{2\_3.yaml}.} \\

На вход поступает строка из слов из 0 и 1, которые разделены пробелом. Обрабатываем по два слова: 0 заменяем на x, 1 заменяем на w, и затем сравниваем слова по длине. Если первое слово больше, то стираем его, а во втором слове обратно заменяем x на 0, w на 1. Если второе слово больше, то перемещаем первое слово на место второго и удаляем лишние символы.






















\section{Квантовые вычисления}

Для выполнения заданий по квантовым вычислениям требуется QDK. Его можно скачать здесь: \url{https://docs.microsoft.com/en-us/azure/quantum/install-overview-qdk}. 
\\\\
Но можно использовать любой пакет, типа \url{https://qiskit.org/}. 
\\\\
В качестве решения задачи надо предоставить схему алгоритма для частного случая при фиксированном количестве кубитов и фиксированных состояниях. 






\subsection{Генерация суперпозиций 1 (1 балл)}

Дано $N$ кубитов ($1 \le N \le 8$) в нулевом состоянии $\Ket{0\dots0}$. Также дана некоторая последовательность битов, которое задаёт ненулевое базисное состояние размера $N$. Задача получить суперпозицию нулевого состояния и заданного.

$$\Ket{S} = \frac{1}{\sqrt2}(\Ket{0\dots0} +\Ket{\psi})$$

То есть требуется реализовать операцию, которая принимает на вход:

\begin{enumerate}
    \item Массив кубитов $q_s$
    \item Массив битов $bits$ описывающих некоторое состояние $\Ket{\psi}$. Это массив имеет тот же самый размер, что и $qs$. Первый элемент этого массива равен $1$.
\end{enumerate}

Так как первые кубиты векторов различны, то применяем оператор Адамара к первому кубиту

Далее если bits[i] равно 1, то запутываем i-ый кубит (оператор CNOT)


\begin{lstlisting}
namespace Solution {
    open Microsoft.Quantum.Primitive;
    open Microsoft.Quantum.Canon;
    operation Solve (qs : Qubit[], bits : Bool[]) : ()
    {
        body
        {
            H(qs[0]);
            for (i in 1..Length(qs) - 1)
                if (bits[i]) 
                    CNOT(qs[0], qs[i]);
        }
    }
}
\end{lstlisting}







\subsection{Различение состояний 1 (1 балл)}

Дано $N$ кубитов ($1 \le N \le 8$), которые могут быть в одном из двух состояний:

$$\Ket{GHZ} = \frac{1}{\sqrt2}(\Ket{0\dots0} +\Ket{1\dots1})$$
$$\Ket{W} = \frac{1}{\sqrt N}(\Ket{10\dots00}+\Ket{01\dots00} + \dots +\Ket{00\dots01})$$

Требуется выполнить необходимые преобразования, чтобы точно различить эти два состояния. Возвращать $0$, если первое состояние и 1, если второе. 

\\\\
\begin{lstlisting}
namespace Solution {
    open Microsoft.Quantum.Primitive;
    open Microsoft.Quantum.Canon;
    operation Solve (qs : Qubit[]) : Int
    {
        body
        {
            mutable countOnes = 0;
            for (i in 0..Length(qs) - 1) {
                if (M(qs[i]) == One) {
                    set countOnes = countOnes + 1;
                }
            }
            if (countOnes == Length(qs) or countOnes == 0)
            {
                return 0;
            }
            return 1;
        }
    }
}


\end{lstlisting}


















\subsection{Различение состояний 2 (2 балла)}

Дано $2$ кубита, которые могут быть в одном из двух состояний:

$$\Ket{S_0} = \frac{1}{2}(\Ket{00} + \Ket{01} + \Ket{10} + \Ket{11})$$
$$\Ket{S_1} = \frac{1}{2}(\Ket{00} - \Ket{01} + \Ket{10} - \Ket{11})$$
$$\Ket{S_2} = \frac{1}{2}(\Ket{00} + \Ket{01} - \Ket{10} - \Ket{11})$$
$$\Ket{S_3} = \frac{1}{2}(\Ket{00} - \Ket{01} - \Ket{10} + \Ket{11})$$


Требуется выполнить необходимые преобразования, чтобы точно различить эти четыре состояния. Возвращать требуется индекс состояния (от $0$ до $3$). 
\\\\
Заготовка для кода:
\begin{lstlisting}
namespace Solution {
        open Microsoft.Quantum.Primitive;
        open Microsoft.Quantum.Canon;
        operation Solve (qs : Qubit[]) : Int
        {
            body
            {

                return 
            }
        }
}
\end{lstlisting}






\subsection{Написание оракула 1 (2 балла)}

Требуется реализовать квантовый оракул на $N$ кубитах ($1 \le N \le 8$), который реализует следующую функцию: $f(\pmb{x}) = (\pmb{b}\pmb{x}) \mod 2$, где  $\pmb{b} \in \{0,1\}^N$ вектор битов и  $\pmb{x}$ вектор кубитов. Выход функции записать в кубит $\pmb{y}$. Количество кубитов $N$ ($1 \le N \le 8$). 
\\\\
Заготовка для кода:
\begin{lstlisting}
namespace Solution {
        open Microsoft.Quantum.Primitive;
        open Microsoft.Quantum.Canon;
        operation Solve (x : Qubit[], y : Qubit, b : Int[]) : ()
        {
            body
            {

            }
        }
}
\end{lstlisting}

\end{document}
